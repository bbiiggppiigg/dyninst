\section{Introduction}
\index[terms]{abc}\index[terms]{abc}\index[terms]{abc}

The normal cycle of developing a program is to edit the source code, compile it, and then execute the resulting binary.  However, sometimes this cycle can be too restrictive.  We may wish to change the program while it is executing or after it has been linked, thus avoiding the process of  re-compiling, re-linking, or even re-executing the program to change the binary.  At first, this may seem like a bizarre goal, however, there are several practical reasons why we may wish to have such a system.  For example, if we are measuring the performance of a program and discover a performance problem, it might be necessary to insert additional instrumentation into the program to understand the problem.  Another application is performance steering; for large simulations, computational scientists often find it advantageous to be able to make modifications to the code and data while the simulation is executing.\\


This document describes an Application Program Interface (API) to permit the insertion of code into a computer application that is either running or on disk.  The API for inserting code into a running application, called dynamic instrumentation, shares much of the same structure as the API for inserting code into an executable file or library, known as static instrumentation.  The API also permits changing or removing subroutine calls from the application program.  Binary code changes are useful to support a variety of applications including debugging, performance monitoring, and to support composing applications out of existing packages.  The goal of this API is to provide a ma-chine independent interface to permit the creation of tools and applications that use runtime and static code patching.  The API and a simple test application are described in \cite{api-code-patching}.  This API is based on the idea of dynamic instrumentation described in [3].

The key features of this interface are the abilities to:
\begin{itemize}
	\item 	Insert and change instrumentation in a running program.
	\item Insert instrumentation into a binary on disk and write a new copy of that binary back to disk.
	\item Perform static and dynamic analysis on binaries and processes.
\end{itemize}


The goal of this API is to keep the interface small and easy to understand.  At the same time, it needs to be sufficiently expressive to be useful for a variety of applications.  We accomplished this goal by providing a simple set of abstractions and a way to specify which code to insert into the application . \footnote{To generate more complex code, extra (initially un-called) subroutines can be linked into the application program, and calls to these subroutines can be inserted at runtime via this interface.
}

\pagebreak